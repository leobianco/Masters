\documentclass[../../main.tex]{subfiles} % Allows individual compilation
\graphicspath{{\subfix{./images/}}}

\begin{document}
	
\section{Schema of our interest}

To explain the aim of this report, recall the results of the seminal paper 
\cite{rohe_spectral_2011}.
\begin{dmath}
	L \rightarrow \mathscr L_{\text{sym}} \xrightarrow{\text{Sp.Cl.}} Z
\end{dmath}
The idea is that the eigenvectors of an observed graph Laplacian \(L\) converge 
to a rotated version of the eigenvectors of the \textit{population} Laplacian 
\(\mathscr L_{\text{sym}}\). In a sense this tells us that all Laplacians of 
observations coming from a fixed model are essentially \(\mathscr 
L_{\text{sym}}\). Then, it is proved that applying the spectral clustering 
algorithm to the population Laplacian of an SBM model with \(k\) communities 
provides precisely the true community assignments \(Z\). This happens because 
the matrix \(Q\), which is the matrix with eigenvectors of 
\(\mathscr{L}_{\text{sym}}\) as columns, contains only \(k\) distinct rows 
which correspond to the community structure (see Lemma 3.1. of 
\cite{rohe_spectral_2011}).

Our point of view is to see the recovery of \(Z\) as the result of the 
maximization of the expected ELBO:
\begin{dmath}
	Z = \argmax_\tau \mathbb{E} \left[\text{ELBO}\left(\tau, 
	\gamma^*\right)\right],
\end{dmath}
where \(\gamma^*\) is the true parameter of the underlying SBM. If the equality 
above is true, then one can use classical theory to show that the maximizer 
\(\hat \tau\) of the ELBO (without the expectation) converges to \(Z\).

\section{Spectral structure of an SBM}

\subsection{Expected Laplacian}

An important object is the expected Laplacian of an SBM, since one could expect 
for the observed Laplacian to converge to it in some sense as the size of the 
SBM grows.

\subsubsection{Expected unnormalized Laplacian} Consider the expected 
unnormalized 
Laplacian, where the expectation is taken conditionally on the model parameters 
and the cluster assignments. Denote it by \(\mathscr L \coloneqq \mathbb E [L 
\vert Z]\). Since \(A_{ij}  \sim \text{Ber}\left(\gamma_{\Omega_i 
\Omega_j}\right)\), it follows that \(\mathbb{E}\left[A_{ij}\right] = 
\gamma_{\Omega_i \Omega_j}\). Here \(\Omega_i\) denotes the community that node 
\(i\) belongs to. Reorder the rows and columns such that the first \(n_1\) rows 
and columns correspond to the first community, the subsequent \(n_2\) rows and 
columns correspond to the second community, and so on. Then, a straightforward 
calculation reveals that
\begin{dmath*} \label{eq:expected-unn-laplacian}
	\mathscr L =
	\begin{pmatrix}
		\mathscr L_1 & -\gamma_{12} \ones[n_1 \times n_2] & \dots & - 
		\gamma_{1k} \ones[n_1 \times n_k] \\
		 & \mathscr L_2 & & \vdots \\
		 & & \ddots & \\
		 & & & \mathscr L_k
		
	\end{pmatrix},
\end{dmath*}
where the entries in the diagonal are the expected Laplacians of each 
community. They can be written as
\begin{align*}
	\mathbf{n} &\coloneqq (n_1, \dots, n_k), \\
	\bar d_i &\coloneqq \langle \Gamma_{i \cdot}, \mathbf{n} \rangle - 
	\gamma_{ii}\\
	\mathscr L_i &= \bar d_i I_{n_i} - \gamma_{ii} J_{n_i}. 
\end{align*}
Notice that \(\bar d_i\) is the expected degree of community \(i\). Also notice 
that after reordering, a similar formula holds for the \textit{observed} 
Laplacian by changing the expected Laplacians on the diagonal by the observed 
ones and the \(\gamma_{ij}\) by the actual adjacency matrices between 
communities \(i\) and \(j\). Finally, notice that reordering in this case does 
not change the eigenvalues of the matrix (as it reorders both rows and columns 
consistently), and only changes the eigenvectors by the same reordering of 
their entries.

\subsubsection{Expected normalized Laplacian}
Consider now the expected \textit{normalized} Laplacian. Remember this 
Laplacian is defined as \(L^{\text{sym}} \coloneqq D^{-1/2} A D^{-1/2}\). An 
equation similar to \ref{eq:expected-unn-laplacian} holds:
\begin{dmath*} \label{eq:expected-norm-laplacian}
	\mathscr L^{\text{sym}} =
	\begin{pmatrix}
		\mathscr L_1^{\text{sym}} & \frac{\gamma_{12}}{\sqrt{\bar d_1 \bar 
		d_2}} \ones[n_1 \times n_2] & \dots & \frac{\gamma_{1k}}{\sqrt{\bar d_1 
		\bar d_k}} \ones[n_1 \times n_k] \\
		& \mathscr L_2^{\text{sym}} & & \vdots \\
		& & \ddots & \\
		& & & \mathscr L_k^{\text{sym}}
	\end{pmatrix},
\end{dmath*}
where the Laplacian of each community might be now computed by
\begin{equation*}
	\mathscr L_i^{\text{sym}} = \frac{\gamma_{ii} J_{n_i}}{\bar d_i}
\end{equation*}

\paragraph{Eigenvectors of \(\mathscr L\).} One can easily verify that the 
following vectors are eigenvectors of \(\mathscr L\):
\begin{align*}
		v_1 = \ones &\implies \lambda_1 = 0 \\
		v_2 = 
			\begin{pmatrix}
				-\frac{n_2}{n_1} \ones[n_1] \\
				\ones[n_2]
			\end{pmatrix} &\implies \lambda_2 = n \gamma_{12}.
\end{align*}
\todo{What are the eigenvectors of these matrices ?}
\\
\todo{Write Rohe's Lemma 3.1, stating that in the eigenvector matrix of 
\(\mathscr L\) there are only \(k\) different rows. This is an important 
characterization of the spectrum of \(\mathscr L\) for spectral clustering.}

\end{document}
