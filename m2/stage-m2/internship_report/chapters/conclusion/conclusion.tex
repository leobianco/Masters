\documentclass[../../main.tex]{subfiles} % Allows individual compilation
\graphicspath{{\subfix{./images/}}}

\begin{document}

In this report, the problem of identifying particular groups of nodes on a graph, 
called \textit{communities}, was studied from a variety of angles. First, methods 
originating from a statistical point of view were constructed: one posits a model for 
the observed graph, then proceeds to infer this model's parameters. Here, the 
particular case of the Stochastic Block Model as a model and the Variational EM as 
inference algorithm were developed. Then, moving to a different standpoint, 
methods based on the spectral analysis of certain graph operators were studied. 
These methods generally approach different types of cut problems on graphs. They 
deal with a relatively simpler mathematical formalism, and are very popular. Once 
these methods were explained, the focus became on understanding how to quantify 
their quality, and present two arguments showing that these methods indeed 
(asymptotically) work when applied to graphs arising as observations of the SBM.

Here are some possible directions of future research. First, consider the problem of 
understanding the \textit{robustness} of these algorithms. While they have an 
extensive theoretical literature studying them, it is known that they may fail to work 
in a variety of situations, be it in applications on real graphs, be it by theoretically 
perturbing the model. Recently, \cite{acharya__2022}, has proposed to study the 
robust estimation of the parameter of an Erd\H{o}s-Rényi model that is ``attacked'' 
by an adversary capable of modifying an \(\varepsilon\)-portion of edges. Also 
recently, \cite{liu__2022} have extended this type of analysis to the SBM. Thus one 
line of research in robustness is that of adversarial attacks.

Another line of research develops spectral methods robust to perturbations of the 
sparse SBM. It was not very understood why spectral algorithms failed in the sparse 
SBM or how to remedy this, until the development of spectral methods based on the 
non-backtracking matrix \cite{abbe_community_2017}. There is now a large 
literature on spectral methods for the sparse SBM. However, it was noted in 
\cite{abbe_graph_2020} that mixing the SBM with a random geometric graph model 
breaks these methods due to the introduction of local loops in the locally tree-like 
structure of the SBM. Thus this is a different direction on the research of robust 
methods for community detection.

A different possible line of research is that of model selection. The two algorithms 
described in this report, and many others, require knowledge of the number \(k\) of 
communities beforehand. However, in practice, when one 
observes a graph this quantity is not known and thus these methods cannot be 
applied. Naturally, the search for methods of estimating the number of communities 
is an important subfield of community detection. In 
\cite{wang_likelihood-based_2016}, a probabilistic approach is proposed, based on 
the optimization of a BIC-like criterion. More recently, \cite{le_estimating_2019} 
propose a spectral approach for the estimation of the number of communities. 
However, there is still much ground to be covered in order to answer this question.

\end{document}
