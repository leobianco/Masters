\documentclass[../../main.tex]{subfiles} % Allows individual compilation
\graphicspath{{\subfix{./images/}}}

\begin{document}

\section{Asymptotic topologies}

\subsection{The case of the Erd\H{o}s-Rényi model}

The first random graph model was the Erd\H{o}s-Rényi model, denoted \(G(n, 
p)\), over graphs with \(n\) vertices \cite{erdos59a}. Under this model, the 
presence of an edge between each pair of nodes is determined by a Bernoulli 
random variable of parameter \(p\). See Figure \ref{fig:obs-erdos-renyi}. 

\begin{figure}[h]
	\centering
	\ifthenelse{\boolean{addtikz}}{
		\subfile{./images/tikz/erdos-renyi/erdos-renyi.tex}
	}{
		\includegraphics{example-image-a}
	}
	\caption{An observation from an Erd\H{o}s-Rényi model \(G(30, 0.2)\)}
	\label{fig:obs-erdos-renyi}
\end{figure}

Although this model does not present clusters of nodes, it is nevertheless of 
great interest, since it reveals a key phenomenon: there exist tightly defined 
and distinct ``asymptotic topologies'' for random graphs arising from this 
model as \(n \to \infty\). Which one arises is a function of the growth of 
\(p\) with respect to \(n\).

\begin{theorem}
	content...
\end{theorem}

\begin{figure}
	\centering
	\begin{subfigure}{.55\textwidth}
		\centering
		\ifthenelse{\boolean{addtikz}}{
			\resizebox{!}{\linewidth}{
				\subfile{./images/tikz/er-giant-0/er-giant-0.tex}
			}
		}{
			\includegraphics[width=\linewidth]{example-image-a}
		}
		\caption{}
		\label{fig:er-giant-0}
	\end{subfigure}
	\hfill
	\begin{subfigure}{.55\textwidth}
		\centering
		\ifthenelse{\boolean{addtikz}}{
			\resizebox{!}{\linewidth}{
				\subfile{./images/tikz/er-giant-1/er-giant-1.tex}
			}
		}{
			\includegraphics[width=\linewidth]{example-image-a}
		}	
		\caption{}
		\label{fig:er-giant-1}
	\end{subfigure}
	\caption{(a) At \(np = 0.8 < 1\), there are some small trees of size at 
		most \(O(\log(n))\). (b) At \(np = 1.33 > 1\) a giant component 
		appears, of 
		size \(O(n^{2/3})\).}
	\label{fig:test1}
\end{figure}

\begin{figure}
	\centering
	\ifthenelse{\boolean{addtikz}}{
		\resizebox{!}{.55\textwidth}{
			\subfile{./images/tikz/er-connectivity-0/er-connectivity-0.tex
			}
		}
	}{
		\includegraphics[width=\linewidth]{example-image-c}
	}
	\caption{At \(p = 0.011 < 0.012\) there exists almost surely an 
		isolated vertex, and the graph is disconnected. When \(p = 0.013 > 
		0.012\), isolated vertices disappear almost surely, and the graph 
		finally becomes connected.}
	\label{fig:er-connectivity-0}
\end{figure}

In the SBM, essentially the same phenomenon happens, and this affects directly 
the performance of algorithms, and has led to ideas of how to increase their 
robustness.

\subsection{The case of the SBM}

\subsection{Why does this matter?}

\end{document}
