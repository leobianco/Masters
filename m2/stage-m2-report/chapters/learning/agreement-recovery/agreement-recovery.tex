\documentclass[../../main.tex]{subfiles} % Allows individual compilation
\graphicspath{{\subfix{./images/}}}

\begin{document}

\section{Agreement and degrees of recovery}

\subsection{Agreement}

Consider the task of evaluating the answer returned by any algorithm for 
community detection. Assume knowledge of the true vector of community 
assignments \(Z^\star\). Given an estimate \(\hat Z\) for \(Z^\star\) how can 
one measure the quality of such an estimation? The most intuitive metric for 
this is the \textit{agreement}, which is simply the average number of nodes 
whose communities were estimated correctly, up to an arbitrary relabeling of 
the communities. This relabeling must be taken into account since the choice of 
the integer associated to a community is arbitrary.

\begin{definition}[Agreement] \label{def:agreement}
	Let \(Z^\star\) and \(\hat Z\) be, respectively, the true and an 
	arbitrary vector of community assignments. Let also \(S_k\) denote the 
	group of permutations of \([k]\). Define the \textit{agreement} between 
	\(Z^\star\) and \(\hat Z\) to be
	\begin{equation} \label{eq:def-agreement}
		A(Z^\star, \hat Z) \coloneqq \frac{1}{n} \max_{\sigma \in S_k} 
		\sum_{i=1}^n \mathbf{1}(Z^\star_i = \sigma \hat Z_i).
	\end{equation}
\end{definition}

However, when studying weaker forms of recovery, under general (asymmetric) 
SBMs, a \textit{normalized} version of this metric is actually needed.

\begin{definition}[Normalized agreement] \label{def:normalized-agreement}
	Let \(Z^\star\) and \(\hat Z\) be, respectively, the true and an 
	arbitrary vector of community assignments. Let also \(S_k\) denote the 
	group of permutations of \([k]\). Define the \textit{normalized agreement} 
	between \(Z^\star\) and \(\hat Z\) to be
	\begin{equation} \label{eq:def-normalized-agreement}
		\tilde A (Z^\star, \hat Z) \coloneqq 
		\frac{1}{k} \max_{\sigma \in S_k} 
		\sum_{k=1}^K \frac{\sum_{i=1}^n \mathbf{1}(Z^\star_i = \sigma 
			\hat Z_i) \mathbf{1} (Z^\star_i = k)}{\sum_{i=1}^n 
			\mathbf{1} (Z^\star_i = k)}.
	\end{equation}
\end{definition}

\subsection{Degrees of recovery}

Definition \ref{def:agreement} can be used to measure different degrees of 
performance of algorithms whose task is to estimate the communities 
\(Z^\star\). The degree to which an algorithm is capable of recovering the 
communities is captured in what are called the different (asymptotic) 
\textit{degrees of recovery}. Notice these are all defined asymptotically.

\begin{definition}[Degrees of recovery]
	Let \((Z^\star, G) \sim\) \SBM{n}{\pi^\star}{\Gamma^\star}, and \(\hat Z\) 
	be the 
	output of an algorithm taking \((G, \pi^\star, \Gamma^\star)\) as input. 
	Then, the following \textit{degrees of recovery} are said to be 
	\textit{solved} if, asymptotically on \(n\), one has
	\begin{itemize}
		\item Exact recovery \(\leftrightarrow\) \(\mathbb P (A(Z^\star, \hat 
		Z) = 1) = 1 - o(1)\),
		\item Almost exact recovery \(\leftrightarrow\) \(\mathbb P (A(Z^\star, 
		\hat Z) = 1 - o(1)) = 1 - o(1)\), 
		\item Partial recovery \(\leftrightarrow\) \(\mathbb P (\tilde 
		A(Z^\star, \hat 
		Z) \geq \alpha) = 1 - o(1), \alpha \in (1/k, 1)\),
		\item Weak recovery (also called detection) \(\leftrightarrow\) 
		\(\mathbb P (\tilde A (Z^\star, \hat 
		Z) \geq 1/k + \Omega(1)) = 1 - o(1)\).
	\end{itemize}
\end{definition}

\begin{remark}
	There is an intuition for why \(\alpha > 1/k\) in the definitions of 
	partial and weak recovery above. If one assumes knowledge of the true 
	parameters \((\pi^\star, \Gamma^\star)\) of the model when designing an 
	estimation algorithm, then the trivial algorithm of simply assigning each 
	node a random community according to \(\pi^\star\) will achieve an 
	agreement of \(\Vert \pi \Vert_2^2\), by the law of large numbers. In 
	particular, in the case where the communities are uniform, \(\pi = 
	\mathbf{1}_k / k\) and the trivial agreement reached will be \(A = 1/k\).
\end{remark}

\end{document}
