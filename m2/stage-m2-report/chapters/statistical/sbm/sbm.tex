\documentclass[../../main.tex]{subfiles} % Allows individual compilation
\graphicspath{{\subfix{./images/}}}

\begin{document}

\section{The stochastic block model}

\subsection{The general SBM}
The canonical probabilistic model for graphs with community structure is called 
the 
stochastic blockmodel, or SBM for short. For a lengthier discussion on the 
origins and variants of this model, refer to \cite{abbe_community_2017}. 

\begin{definition}[Stochastic blockmodel] \label{def:sbm}
	Let \(n \in \mathbb N\), \(k \in \mathbb N\), \(\pi = (\pi_1, \dots, 
	\pi_k)\) be a probability vector on \([k] := \{1, \dots, k\}\) and  
	\(\Gamma\) be a \(k \times k\) symmetric matrix with entries \(\gamma_{ij} 
	\in [0, 1]\). A pair \((Z, G)\) is said to be \textit{drawn under a 
	\SBM{n}{\pi}{\Gamma}} if 
	\begin{itemize}
		\item \(Z = (Z_1, \dots, Z_n)\) is an \(n\)-tuple of \(\mathbb 
		N^k\)-valued random variables \(Z_i \sim \mathcal M (1, \pi)\),
		\item \(G\) is a simple graph with \(n\) vertices whose symmetric 
		adjacency matrix has zero diagonal and for \(j > i, A_{ij} \vert Z \sim 
		\text{Ber}(\gamma_{Z_i, Z_j})\), the lower triangular part being 
		completed by symmetry.
	\end{itemize}
\end{definition}

\begin{figure}[ht]
	\centering
	\ifthenelse{\boolean{addtikz}}{
		\subfile{./images/tikz/sbm-obs/sbm-obs.tex}
	}{
		\includegraphics{example-image-a}
	}
	\caption{An example of an SBM graph with assortative communities}
	\label{fig:sbm-example}
\end{figure}

\begin{remark}
	The quantity \(n\) should be thought of as being the number of nodes in 
	\(G\), \(k\) should be thought of as being the number of communities in 
	\(G\), \(\pi\) should be thought of as being a prior on the community 
	assignments \(Z\), and \(\Gamma\) should be thought of as a matrix of 
	intra-cluster and inter-cluster connectivities. The random variables of the 
	model are the \(n\) community assignments \(Z\) and the \(\binom{n}{2}\) 
	entries \(A_{ij}\) of the adjacency matrix.
\end{remark}

\begin{remark}
	Although each community assignment \(Z_i\) is a vector, the same \(Z_i\) 
	can be used to denote the \textit{number} of the community that node \(i\) 
	is assigned to, to make notation lighter.
\end{remark}

It is important to emphasize that although intuition frequently refers to the 
assortative case, such as in Figure \ref{fig:sbm-example}, the SBM is versatile 
and can reproduce many other characteristics of graphs with communities. For 
instance, the SBM can generate bipartite graphs as a model for the example 
given in the introduction, where couples dance in a party; it can also generate 
graphs with ``stars'', and reproduce the ``core-periphery'' phenomenon. See 
Figure \ref{fig:sbm-versatile}.


\begin{figure}
	\centering
	\setlength\tabcolsep{4ex}
	\makebox[\textwidth][c]{
	\begin{tabular}{cc}
	\begin{subfigure}{.45\textwidth}
		\centering
		\ifthenelse{\boolean{addtikz}}{
			\resizebox{!}{\linewidth}{
				\subfile{./images/tikz/bipartite-sbm/bipartite-sbm.tex}
			}
		}{
			\includegraphics[width=\linewidth]{example-image-a}
		}
		\caption{}
		\label{fig:bipartite-sbm}
	\end{subfigure}
	&
	\begin{subfigure}{.45\textwidth}
		\centering
		\ifthenelse{\boolean{addtikz}}{
			\resizebox{!}{\linewidth}{
				\subfile{./images/tikz/star-sbm/star-sbm.tex}
			}
		}{
			\includegraphics[width=\linewidth]{example-image-a}
		}
		\caption{}
		\label{fig:star-sbm}
	\end{subfigure}
	\\
	\multicolumn{2}{c}{
	\begin{subfigure}{.5\textwidth}
		\centering
		\ifthenelse{\boolean{addtikz}}{
			\resizebox{!}{\linewidth}{
				\subfile{./images/tikz/core-periphery-sbm/core-periphery-sbm.tex}
			}
		}{
			\includegraphics[width=\linewidth]{example-image-a}
		}
		\caption{}
		\label{fig:core-periphery-sbm}
	\end{subfigure}
	}
	\end{tabular}
	}
	\caption{The SBM is versatile and can give rise to different features, such 
	as (a) bipartite structures, (b) star structures, (c) core-periphery 
	structures.}
	\label{fig:sbm-versatile}
\end{figure}

\subsection{The symmetric SBM}
Even though the SBM is a simple and intuitive model for graphs with 
communities, the calculations associated with it can already yield long 
expressions and present subtleties. For this reason, it is desirable to 
have a yet simpler version of the SBM where one can test intuitions and perform 
preliminary calculations. The symmetric SBM is precisely such a model. 
\begin{definition}[Symmetric SBM]
	The pair \((X, G)\) is drawn from \SSBM{n}{k}{p}{q} if it is drawn from an 
	SBM model with \(\pi = \frac{1}{k} \ones[k]\) and \(\Gamma\) taking values 
	\(p\) on the diagonal and \(q\) outside the diagonal.
\end{definition}

\end{document}