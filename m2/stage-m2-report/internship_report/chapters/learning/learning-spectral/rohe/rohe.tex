\documentclass[../../main.tex]{subfiles} % Allows individual compilation
\graphicspath{{\subfix{./images/}}}

\begin{document}
	\section{The case of of spectral clustering}  % 📝
	\subsection{Motivation}  % ✅
	The work of \cite{rohe_spectral_2011} studies the question of whether 
	spectral clustering is capable of recovering the communities of graphs 
	generated under an SBM. Although they 
	do not imply consistency of the 
	algorithm and have quite strong hypotheses, their results became very 
	popular for the intuitions they give. The objective of this section is to explain 
	them and to understand a bit better when and \textit{why} the spectral 
	clustering algorithm might work or not on graphs generated from an SBM. 
	Some details will be given in the appendices, but the main objective is to 
	make the ideas clear.
	
	\subsection{Notations}  % ✅
	The matrix \(L = D^{-1/2} A D^{-1/2}\) is used as Laplacian. At first 
	sight, it may seem different from the normalized Laplacian \(L_{\text{sym}}\) 
	defined in \eqref{def:laplacian-matrices}. However, one can directly check 
	these matrices have the same eigenvectors: if \(L_{\text{sym}} v = \lambda 
	v\), then \(L v = (1 - \lambda) v\), and if \(L v = \lambda v\), then 
	\(L_{\text{sym}} v = (1 - \lambda) v\). The difference of eigenvalues simply 
	implies that instead of taking the \(k\) smallest eigenvectors one must take 
	the \(k\) largest; this being taken into account, these matrices are equivalent 
	for the analysis. 
	\begin{remark}
		Do pay attention when translating these results to the unnormalized 
		Laplacian, since its eigenvectors \textit{will not} be equal to those of \(L\) 
		(but they can easily be transformed using \(D^{1/2}\)).
	\end{remark}

	The \textit{expected} adjacency matrix is defined as \(\mathscr A \coloneqq 
	\mathbb E [A \vert Z^{\star}]\), and the \textit{expected} degree matrix is 
	the 
	diagonal matrix with entries \(\mathscr D_{ii} = \sum_k \mathscr A_{ik}\). 
	Naturally, the expected Laplacian is defined as \(\mathscr L = \mathscr 
	D^{-1/2} \mathscr A \mathscr D^{-1/2}\). A quantity that will play a 
	fundamental role is defined as \(\rho_n 
	\coloneqq \min_{i=1, \dots, n} \mathscr D_{ii}^{(n)}/n\). Intuitively, it 
	measures how quickly the number of edges accumulates as the number of 
	nodes in the graphs generated grows. Notice that \(D_{ii}\) is the 
	\textit{expected degree} of community \(i\).
	
	The structure of the results is diagrammatically shown in Figure 
	\ref{fig:tikz-diagram-rohe}, and the analysis is divided in two parts. First, 
	consider the question of whether the empirical counterpart of the expected 
	Laplacian approaches it as the size of the graphs observed grows. This could 
	be hoped for, since in the case of the adjacency matrix, the entries of \(A 
	\vert Z\) are independent (technically, its upper triangular part is 
	independent, and the lower triangular part is identical to it), thus there is a 
	strong concentration of \(A\) around its expected version. As it will be seen, 
	it is not as straightforward for the Laplacian, but fortunately with some 
	additional work it is possible to show a similar concentration for it. Second, 
	there is the question of whether the expected Laplacian contains or not the 
	information on the communities. If it does, and given that the empirical 
	Laplacian approaches it, then one could try to estimate this information from 
	the empirical version.
	\begin{figure}
		\ifthenelse{\boolean{addtikz}}{
			\subfile{./images/tikz/diagram_rohe/diagram_rohe.tex}
		}{
			\includegraphics{example-image-a}
		}
		\caption{Diagram of results in \cite{rohe_spectral_2011}. First, the 
		question of whether the empirical Laplacians approach the expected 
		version or not is studied; then, the question of whether the expected 
		Laplacian contains or not the complete information on the communities is 
		analyzed.}
		\label{fig:tikz-diagram-rohe}
	\end{figure}	
	
	\subsection{Convergence of eigenvectors}  % ✅
	The first part consists in showing a 
	preliminary result of concentration of 
	the 
	empirical Laplacian towards its expectation, that is, showing that the 
	eigenvectors of the empirical Laplacian converge in some sense to the 
	eigenvectors of the expected Laplacian. This conclusion would be immediate 
	if 
	the former converged to the latter in Frobenius norm, i.e. if \(\Vert L^{(n)} - 
	\mathscr L^{(n)} \Vert_F \to 0\), since then the \textit{Davis-Kahan} theorem 
	\cite{davis-kahan} would imply the alignment of their eigenspaces. 
	Unfortunately, these matrices do 
	not converge in Frobenius norm \cite{rohe_spectral_2011}, so a ``detour'' is 
	needed in order to achieve this 
	convergence. 
	
	What is demonstrated is that, under certain conditions, their ``squares'' converge 
	in Frobenius norm.
	\begin{proposition}
		If there exists some \(N > 0\) such that \(\rho_n^2 \log n > 2\) for all \(n 
		> 
		N\), then
		\begin{equation}
			\Vert L^{(n)} L^{(n)} - \mathscr L^{(n)} \mathscr L^{(n)} \Vert_F = 
			o \left( \frac{\log n}{\rho_n^2 n^{1/2}}\right) \quad \text{a.s.}
		\end{equation}
	\end{proposition}

	It is important to interpret the hypothesis on \(\rho_n\). It is a type of 
	hypothesis 
	commonly present in consistency results such as this one. It sets the growth 
	regime for the expected degree in order for the result to hold. As seen in 
	\ref{sec:asymptotic-topologies}, this type of hypothesis is related to the 
	asymptotic topology of the graphs arising from the model. In a sense, the 
	strictness imposed by this hypothesis can be used to measure the ``strength'' of 
	a given consistency result. The stronger this hypothesis, the denser the graphs 
	must be for the result to hold, and the less robust the proof will be to sparsity.
	
	A version of the Davis-Kahan theorem then implies that, under some conditions, 
	the eigenvectors of \(L^{(n)} L^{(n)}\) converge to the eigenvectors of \(\mathscr 
	L^{(n)} \mathscr L^{(n)}\), up to some (unknown) \textit{rotation}. This in turn can 
	be transformed into a convergence statement for the eigenvectors of the 
	Laplacian without the square. The precise statement of the theorem is quite long 
	(due to an hypothesis dealing with the rate with which the eigengap closes; 
	the eigengap is the minimal distance between an eigenvalue of the 
	eigenvectors being used in the algorithm, and the rest of the eigenvalues);  
	see \cite{rohe_spectral_2011} for the details and a complete statement.
	\begin{proposition}
		If there exists \(N > 0\) such that \(\rho_n^2 \log n > 2\) for all \(n > N\) 
		and 
		if the gap \(\delta_n\) between the eigenvalues of interest and the other 
		eigenvalues does not go to zero too quickly (cf. \cite{rohe_spectral_2011}), 
		then for some sequence of orthonormal rotations \(O_n\),
		\begin{equation}
			\Vert X_n - \mathscr X_n O_n \Vert_F = 
			o \left( \frac{\log n}{\delta_n \rho_n^2 n^{1/2}}\right) \quad \text{a.s.},
		\end{equation}
		where \(X_n\) is the orthogonal matrix with the eigenvectors of \(L^{(n)}\) as 
		columns; similarly for \(\mathscr X_n\) and \(\mathscr L ^{(n)}\).
		\label{prop:convergence-eigenvectors}
	\end{proposition}

	In short, the convergence of eigenvectors takes place up to a rotation. This 
	rotation is immaterial for the spectral clustering algorithm, since the 
	\(k\)-means will cluster the empirical eigenvectors solely based on their 
	relative concentration.
	
	\subsection{Bounding the number of misclassified nodes}  % 📝
	The second part of the analysis is showing that the estimation 
	\(\hat{Z}_{\text{Spec}}^{\text{(Alg)}}\) obtained using the spectral clustering 
	algorithm asymptotically matches the ground truth assignments \(Z^\star\). It 
	starts by showing that applying the algorithm to the expected Laplacian (instead 
	of the observed one) recovers the partitions of the SBM, even non-asymptotically.
	\begin{lemma}
		There exists a matrix \(\mu \in \mathbb R^{k \times k}\) such that the 
		eigenvectors of \(\mathscr L\) corresponding to the nonzero eigenvalues are 
		the columns of \(Z \mu\). Furthermore, it holds that
		\begin{equation} \label{eq:equivalence_rohe}
			z_i \mu = z_j \mu \iff z_i = z_j,
		\end{equation}
		where \(z_i\) is the \(i\)-th row of \(Z\).
		\label{lemma:retrieval-expectation}
	\end{lemma}
	\begin{proof}
		A sketch of the proof is given in Section 
		\ref{appendix:sketch-lemma-retrieval}.
	\end{proof}

	The meaning of equivalence \eqref{eq:equivalence_rohe} is that rows \(i\) 
	and \(j\) of \(Z \mu\) are equal if, and only if, the corresponding rows of 
	\(Z\) are equal, that is if nodes \(i\) and \(j\) belong to the same community. 
	Since there are \(k\) communities, this implies that there can be at most \(k\) 
	unique rows in the matrix \(Z \mu\) of eigenvectors of \(\mathscr L\). 
	Spectral clustering applies \(k\)-means to these vectors, and thus these 
	become precisely the centroids of \(k\)-means (since one is applying 
	\(k\)-means to at most \(k\) different vectors). The rows of \(Z \mu\) will 
	then be attributed to the centroid they are equal to, and by the equivalence 
	in Equation \eqref{eq:equivalence_rohe}, this implies that spectral clustering 
	applied to the expected Laplacian \(\mathscr L\) identifies the communities. 
	The rows \(z_i \mu\) constitute what will be called the \textit{expected} 
	centroids. Of course, the above is a theoretical result, since in practice the 
	expected Laplacian is unknown.
	
	To go from this theoretical consideration to an empirical one, consider that 
	\(k\)-means is applied to the rows of \(X\), the matrix whose columns are 
	the first \(k\) eigenvectors of the empirical Laplacian \(L\). For node \(i\), 
	denote \(c_i\) the \textit{empirical} centroid associated to it by the 
	\(k\)-means algorithm. Intuitively, it is expected that if the algorithm works, 
	then the empirical centroid \(c_i\) is closer to the expected centroid \(z_i 
	\mu\) than to the other \(z_j \mu, j \neq i\) expected centroids, since 
	these represent communities distinct of that of \(i\). However, remember 
	that a rotation \(O\) needed to be included in proposition 
	\ref{prop:convergence-eigenvectors}, and it must also be included here.  The 
	main result in 
	\cite{rohe_spectral_2011} bounds the size of the set \(\mathscr M\) of nodes 
	\(i\) closer to some (rotated) expected centroid \(O z_j \mu, j \neq i\) 
	associated to another community than to its rotated expected centroid \(O 
	z_i \mu\). \cite{rohe_spectral_2011} call the nodes in \(\mathscr M\) 
	``misclassified'' nodes, although this is misleading since it does not make 
	any reference to the estimated labels nor to the ground truth communities.
	\begin{theorem}
		Define \(P_n \coloneqq \max_{j = 1, \dots, k} (Z^t Z)_{jj}\), which is the 
		size of the largest community for graph-sample \(n\). Denote by 
		\(\lambda_n\) the smallest non-zero eigenvalue of \(\mathscr L_n\). If 
		\(n^{-1/2}(\log n)^2 = O(\lambda_n^2)\) and there exists \(N\) such that 
		for all \(n > N, \rho_n^2 > 2/\log n\), then
		\begin{equation}
			\vert \mathscr M \vert = o \left(\frac{P_n (\log n)^2}{\lambda_n^4 
			\rho_n^4 n}\right).
		\end{equation}
	\label{thm:main-rohe}
	\end{theorem}
	
	% ----------------------
	% pName: CLARIFICATIONS ON STRENGTH OF HYP. AND RECOVERY
	Intuitively, this result states that if the smallest eigenvalue of the expected 
	Laplacian does not decrease too fast, and if the expected degree grows fast 
	enough, then the empirical centroids \(c_i\) will be closer the ``correct'' 
	rotated expected centroids \(O z_i \mu\) than to the other \(O z_j \mu\), 
	with a proportion tending to one. Notice however that this result does not 
	imply convergence of centroids to their expected version, and more 
	importantly, \textit{does not} show consistency of spectral clustering. It is 
	important to also emphasize that the conditions imposed by Theorem 
	\ref{thm:main-rohe} are quite strong. For instance, the condition on the 
	degree, \(\rho_n^2 \log n > 2\) is almost as strong as requiring that the 
	expected degrees grow as \(n\). That is, very dense graphs are required, and 
	many real graphs are sparser than this, in a way one cannot expect these 
	results to hold in practice.
	
\end{document}