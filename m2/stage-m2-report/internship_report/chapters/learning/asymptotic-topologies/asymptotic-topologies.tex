\documentclass[../../main.tex]{subfiles} % Allows individual compilation
\graphicspath{{\subfix{./images/}}}

\begin{document}

\section{Asymptotic topologies}  % 📝
\label{sec:asymptotic-topologies}
This section details one factor that might impact algorithms' performance in the 
context of the analysis performed in this chapter: the asymptotic structure of 
the graphs to which they are applied to. First, it is explained how these 
structures appear in models relevant for this report. Then, it is explained how 
this impacts the robustness of algorithms.

\subsection{The case of the Erd\H{o}s-Rényi model}
The first random graph model was the Erd\H{o}s-Rényi model, denoted \(G(n, 
p)\), over graphs with \(n\) vertices \cite{erdos59a}. Under this model, the 
presence of an edge between each pair of nodes is determined by a Bernoulli 
random variable of parameter \(p\). See Figure \ref{fig:obs-erdos-renyi}. 
\begin{figure}[h]
	\centering
	\ifthenelse{\boolean{addtikz}}{
		\subfile{./images/tikz/erdos-renyi/erdos-renyi.tex}
	}{
		\includegraphics{example-image-a}
	}
	\caption{An observation from an Erd\H{o}s-Rényi model \(G(30, 0.2)\)}
	\label{fig:obs-erdos-renyi}
\end{figure}

Although this model does not present clusters of nodes, it is nevertheless of 
great interest, since it reveals a key phenomenon: there exist tightly defined 
and distinct \textit{asymptotic topologies} for random graphs arising from this 
model as \(n \to \infty\). Which one arises is a function of the growth of 
\(p\) with respect to \(n\).

\begin{theorem}
	% ----------------------
	% pName: CYCLE
	
	% ----------------------
	% pName: GIANT COMPONENT
	
	% ----------------------
	% pName: CONNECTIVITY
	
	% ----------------------
\end{theorem}

\begin{figure}
	\centering
	\begin{subfigure}{.55\textwidth}
		\centering
		\ifthenelse{\boolean{addtikz}}{
			\resizebox{!}{\linewidth}{
				\subfile{./images/tikz/er-giant-0/er-giant-0.tex}
			}
		}{
			\includegraphics[width=\linewidth]{example-image-a}
		}
		\caption{}
		\label{fig:er-giant-0}
	\end{subfigure}
	\hfill
	\begin{subfigure}{.55\textwidth}
		\centering
		\ifthenelse{\boolean{addtikz}}{
			\resizebox{!}{\linewidth}{
				\subfile{./images/tikz/er-giant-1/er-giant-1.tex}
			}
		}{
			\includegraphics[width=\linewidth]{example-image-a}
		}	
		\caption{}
		\label{fig:er-giant-1}
	\end{subfigure}
	\caption{(a) At \(np = 0.8 < 1\), there are some small trees of size at 
		most \(O(\log(n))\). (b) At \(np = 1.33 > 1\) a giant component 
		appears, of 
		size \(O(n^{2/3})\).}
	\label{fig:test1}
\end{figure}

\begin{figure}
	\centering
	\ifthenelse{\boolean{addtikz}}{
		\resizebox{!}{.55\textwidth}{
			\subfile{./images/tikz/er-connectivity-0/er-connectivity-0.tex
			}
		}
	}{
		\includegraphics[width=\linewidth]{example-image-c}
	}
	\caption{At \(p = 0.011 < 0.012\) there exists almost surely an 
		isolated vertex, and the graph is disconnected. When \(p = 0.013 > 
		0.012\), isolated vertices disappear almost surely, and the graph 
		finally becomes connected.}
	\label{fig:er-connectivity-0}
\end{figure}

\subsection{The case of the SBM}  % ⭕️
In the SBM, essentially the same phenomenon happens.
\begin{proposition}
	Content.
\end{proposition}

\subsection{Why does this matter?}  % ⭕️
% ----------------------
% pName: EXPLAINING THE INTEREST FOR COMMUNITY DETECTION
% 1. Connect with previous paragraph
This phenomenon is of interest for community detection, because the analysis 
of algorithms takes place asymptotically. Knowing what structures appear can 
provide an intuition to the common hypotheses appearing on consistency 
results regarding the asymptotic regime of the expected degree. It can also 
reveal weaknesses of some methods that rely too heavily on these structures, 
implying a lack of robustness to perturbations.
% 3. Give up to three details
% 4. Brief lead (connector) for next paragraph

% ----------------------

, and this affects directly 
the performance of algorithms, and has led to ideas of how to increase their 
robustness.

\end{document}
