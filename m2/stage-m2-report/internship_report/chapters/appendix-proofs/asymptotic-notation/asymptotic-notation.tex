\documentclass[../../main.tex]{subfiles} % Allows individual compilation
\graphicspath{{\subfix{./images/}}}

\begin{document}

\section{Asymptotic notation}
\label{app:asymptotic-notation}
This section some asymptotic notations for sequences that are commonly used 
in probability. To avoid redundancy, in what follows, let \((x_n)_{n \in \mathbb 
N}\) and \((y_n)_{n \in \mathbb N}\) be a pair of real sequences.

\begin{definition}
	One denotes \(x_n = O (y_n)\) (read \textit{\(x_n\) is big-oh \(y_n\)}) if there 
	exists some \(N \in \mathbb N\) and some real constant \(C > 0\) such that 
	\(\vert x_i \vert \leq C \vert y_i \vert\) for all \(i > N\).	
\end{definition}

\begin{definition}
	One denotes \(x_n = o (y_n)\) (read \textit{\(x_n\) is 
	little-oh \(y_n\)}) if for every \(C > 0\) there exists some \(N \in \mathbb N\) 
	such that for all \(n > N\), \(\vert x_n \vert < C \vert y_n \vert\).
\end{definition}

\begin{definition}
	One denotes \(x_n = \Omega (y_n)\) (read \textit{\(x_n\) is capital-omega 
	\(y_n\)}) if \(y_n = O(x_n)\).	
\end{definition}

\begin{definition}
	One denotes \(x_n = \omega (y_n)\) (read \textit{\(x_n\) is little-omega 
		\(y_n\)}) if \(y_n = o (x_n)\).
\end{definition}

\end{document}
